\begin{flushleft}
    Micro ROS ist eine fork von ROS und damit auch lauffähig auf schwächeren Systemen.
    Als Microcontroller für unsere Roboter Platform fiel die Entscheidung auf einen ESP32.
    Der ESP32 bietet für seine Kosten und seine Größe ausreichend Rechenpower und
    desweiteren unterstützt das Micro-ROS Framework bereits den ESP32.
    Als alternativen für den ESP32 könnten ansonsten noch ein Rapsberry PI oder ein Arduino
    bzw. ATmega Microcontroler eingesetzt werden.

    Für das Front-End kommen viele Frameworks in Frage aber wir wollten unsere Auswahl vorerst auf React oder Vue beschränken.
    Mit diesen Frameworks haben bereits alle unsere Gruppenteilnehmer gearbeitet und somit hatten wir
    alle schon Erfahrung damit gesammelt. Und einsteig wurde erleichtert.

    Als Notlösung für unser Frontend kämme ansonsten Plain HTML und CSS in frage, was aber höchstwahrschienlich
    in einem sehr großen Aufwand enden würde.
\end{flushleft}