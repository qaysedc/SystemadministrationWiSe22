\subsection{Frontend}
    Das Frontend könnte durch normales plain HTML, CSS und JS realisiert werden. 
    Da dies aber nicht mehr dem heutigen technischen standart entspricht kämen viele verschiedene Frontend Frameworks in frage.
    Zur Auswahl stehen React, Vue oder Angular.

\subsection{Roboter}
    Zur Demonstration unserer Orchestrierungssoftware soll ein kleiner Roboter gebaut werden.
    Der Roboter sollte, wie in unseren Anforderungen bereits aufgelistet, einfach und schnell gebaut werden können.
    Und damit unser kleiner Roboter auch problemlos in der Welt navigieren kann muss dieser auch mit einer Art Lenkung ausgestattet werden.

    Für solch einen Steuermechanismus kommen viele verschiedene Lösungen infrage:
    \\
    Ein Kettenantrieb ähnlich wie in Baggern wäre denkbar. Er benötigt nur 2 Motoren und bietet viel Bewegungsfreheit ohne komplizierte Mechanik. 
    \\
    Die Ackermann Lenkung wie Sie heute in allen PKW und LKW vorkommt wäre ebenfalls denkbar. Hierbei müsste man 
    nur einen Motor einbauen für den allgemeinen vortreib und einen kleinen Servo um die tatsächliche Lenkbewegung der
    Vorder- oder Hinterache zu realisieren. 
    \\
    Der Differentialantrieb wäre eine weitere Methode unseren Antrieb zu realisieren, hierbei werden wie beim Kettenantrieb
    ebenfalls zwei Motoren benötigt. Allerdings entfällt hierbei die Notwendigkeit einer Laufkette.
    \\
    Unsere letze Idee wäre eine Knicklenkung wie sie oft in Baustellenfahrzeugen eingesetzt wird. Es bräuchte nur einen Motor
    als Antrieb und die etwas kompliziertere Mechanik von der Ackermann Lenkung könnte durch eine simplere ersetzt werden.
    \\

    Unser Roboter kommt natürlich nicht nur mit einem einfachen Antrieb aus sondern muss diesen auch entsprechend 
    ansteuern können.
    Ein Arduino könnte sehr gut dafür geeignet sein.
    \\
    Ein ESP32 käme auch in frage. Er bietet weitaus mehr Rechenpower als ein Arduino bei minimaler Preiserhöhung.
    Außerdem unterstützen die meisten ESP32 eine Verbindung über WLAN.
    \\
    Als teuerste aber auch Leistungsstärke Kontrolleinheit könnte auch ein Raspebrry Pi dienen.
    
    Da wir unsere Roboter mit hilfe vom ROS Framework ansteuern wollen, sollte unsere gewählte Steuereinheit Micro-ROS unterstützen.
    Micro ROS ist eine fork von ROS selbst und dadurch auchauf schwächeren Systemen lauffähig.