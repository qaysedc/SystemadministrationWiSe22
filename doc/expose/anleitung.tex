\begin{flushleft}
    \label{frontend_install}ROS2 Docker Image mit Foxy Fitzroy Version.
    \begin{lstlisting}[language=bash]
        docker pull osrf/ros:foxy-desktop 
    \end{lstlisting}

    \textit{foxy-desktop} kann hier mit jeder anderen beliebigen Version ausgetauscht werden. Mehr Informationen hierzu findet man im entsprechenden Docker Hub: 
    \begin{lstlisting}
        https://hub.docker.com/r/osrf/ros/
    \end{lstlisting}

    Rosbridge mit apt-Paketmanager installieren:
    \begin{lstlisting}[language=bash]
        sudo apt install ros-<ROS_DISTRO>-rosbridge-server 
    \end{lstlisting}

    Als nächstes muss die Entwicklungsumgebung gesourced werden und anschließend das Launch-File für die \textit{rosbridge} gestartet:
    \begin{lstlisting}
        source /opt/ros/foxy/setup.bash
        
        ros2 launch rosbridge_server rosbridge_websocket_launch.xml
    \end{lstlisting}

    Roslibjs Inkludierung in HTML ohne React:
    \begin{lstlisting}[language=html]
    <script 
        type="text/javascript" 
        src="http://static.robotwebtools.org/roslibjs/current/roslib.min.js">
    </script> 
    \end{lstlisting}

    Verbindung mit rosbridge über roslibjs-Bibliothek:
    \begin{lstlisting}
        const ros = new ROSLIB.Ros({ url: "ws://"+ipAddress+":9090" });
        ros.on("connection", () => {
            // successful connection
        });
        ros.on("error", (error) => {
            // error in connection
        });
    \end{lstlisting}

    Nachdem eine neue React Installation durchgeführt wurde, kann mit diesem Befehl die roslibjs Bibliothek in ein bestehendes Projekt installiert..:
    \begin{lstlisting}[language=bash]
        npm install roslib 
    \end{lstlisting}
    
    ..und inkludiert werden:
    
    \begin{lstlisting}
        import ROSLIB from 'roslib';
    \end{lstlisting}
\end{flushleft}