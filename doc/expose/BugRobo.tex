\begin{flushleft}
        Initial sollten kleine 5V Motoren aus alten DVD-Laufwerken mit selbstgebauten Getriebe als Antrieb dienen.
        Jedoch musste man recht schnell festellen dass das Design von einem Getriebe doch nicht so einfach ist.
        Es muss auf den perfekten Abstand der Zähne der Zahnräder geachtet werden. Dass z.B. die Zahnräder konzentrisch sind und 
        sich leichtgängig drehen lassen. Letztendlich war unser erster Getriebeversuch auch der Letzte.
        Der Motor hatte einfach viel zu wenig Drehmoment um alle auftretenden Reibungsverluste zu überkommen.
        Ein netter Versuch um das ein oder andere zu lernen war er aber dennoch.
        \\
        Der erste Versuch eine Freilaufrolle zu designen und zu drucken funktionierte nicht so gut wie erhofft.
        Die Rolle konnte zwar ohne Probleme gedruckt werden, allerdings konnte sie sich nicht frei und leichtgängig genug drehen.
        Das lag zum einen daran, dass der Rollendurchmesser zu klein gewählt war und der Versatz von der Rolle selbst zu ihrem vertikalen Drehpunkt zu groß war.
        Genau das sorgte für große Reibung und schlussendlich dafür, dass sich die Rolle nicht in Fahrtrichtung ausrichten konnte.
        \\
        Wie bereits in [\ref{sec:electronic}] erwähnt, war geplant den Akku für den Roboter selbst aus 18650 Zellen zu fertigten.
        Allerdings hätte man dafür noch eine Lade- \& Entladeelektronik benötigt und zusätzlich ein Punktschweißgerät.
        Es war einfach zu viel Aufwand für zu wenig Gewinn. Deshalb entschieden wir uns einfach für die 9V Blockbatterie.
        (Großen Dank an Dipl.-Ing. Joachim Feßler für die Bereitstellung der Akkus)

\end{flushleft}