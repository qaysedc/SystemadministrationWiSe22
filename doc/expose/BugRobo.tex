\begin{flushleft}
    Getriebe selber bauen sollte doch kein Problem sein, oder?:
    Initial sollten kleine 5V Motoren aus alten DVD-Laufwerken mit selbstgebauten getriebe als Antrieb dienen.
    Jedoch musste man recht schnell festellen dass das Design von einem Getriebe doch nicht so einfach ist.
    Es muss auf den perfekten Abstand der Zähne der Zahnräder geachtet werden. Das die Zahnräder konzentrisch und 
    sich leichtgängig drehen und noch vieles mehr. Letzendlich war unser erster Getriebeversuch auch der letzte.
    Der Motor hatte einfach viel zu wenig Drehmoment um alle auftrettenden Reibungsverluste zu überkommen.
    Ein netter versuch um das ein oder andere zu lernen war er aber denoch.
   
    (bild folgt)

    Das Problem mit der Freilaufrolle:
    Der erste Versuch eine Freilaufrolle zu designen und zu drucken funktionierte nur zu 50(prozent).
    Die Rolle konnte zwar ohne Probleme gedruckt werden, allerdings konnte sie sich nicht frei und leichtgängig genug drehen.
    Das lag zum einen daran das 

    (bild folgt)

\end{flushleft}