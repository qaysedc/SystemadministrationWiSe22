\begin{flushleft}
    Die Entwicklung mit dem Mikrocontroller findet mit Micro-ROS statt, da das die Standard-Bibliothek für die Entwicklung mit Microcontrollern mit ROS2 ist.\\
    Das Besondere an ROS2 und Micro-ROS im Vergleich zu ROS1 ist, dass es ermöglicht ein Real Time Operating System auf dem Microcontroller auszuführen. Ein RTOS wird standardmäßig mitinstalliert, ist aber für uns auch interessant, da wir bisher noch nicht mit RTOS gearbeitet haben und es eine gute Lernmöglichkeit mit beschränkten Aufwand ist.
    Wie ROS selbst, ist Micro-ROS open-source und kostenlos.\\
    Außerdem gibt es eine bereits bestehende Toolchain namens ESP-IDF, die das Cross-Compilieren, Flashen, Monitoring und Debugging erleichtert.

\end{flushleft}