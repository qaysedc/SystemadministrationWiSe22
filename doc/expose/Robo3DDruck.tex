\begin{flushleft}
    Die Wichtigste Anforderung an die Platform war das diese Modular aufgebaut werden kann. Sprich man kann im Nachhinein
    neue Teile, Module oder komplett andere Sensoren einfach anschrauben aber auch kaputte bzw. alte Teile problemlos erneuern.

    Da es sich beim Differentialantreib nur um eine Mechanische Anordnung der Motoren selbst handelt, musste noch ein
    geeignetes Bauteil als Motor selbst gefunden werden.
    Die Motoren müssen ein recht hohes Drehmoment aber eine kleine Umdrehungszahl pro Minute aufweisen.
    Initial war es deshalb die Idee für normale 5V Motoren wie sie auch in DVD Laufwerken verwendet werden ein kleines
    Getriebe zu konzeptionieren und zu bauen.

    \begin{figure}[h!]
        \centering
        \includegraphics[width=0.3\textwidth]{imgs/Roboter/Real/Getriebe.jpg}
        \caption{Prototyp des eigenbau Motorgetriebe}
        \label{fig:prototyp_transmission}%
    \end{figure}

    Die Idee bewies sich aber recht schnell als zu schwierig und wurde deshalb fallen gelassen.
    Als Ersatz für unseren Misslungenen Versuch fiel die Entscheidung auf Modellbau Getriebemotoren. Siehe Abbildung \ref{fig:robot_motor}.

    \begin{figure}[h!]
        \centering
        \includegraphics[width=0.3\textwidth]{imgs/Roboter/Real/41eJJZ8mOOL._SX342_.jpg}
        \caption{Modelbau Getriebemotor}
        \label{fig:robot_motor}%
    \end{figure}

    Ein Anforderung an unsere Demo Roboter Platform war die 3D-Druckbarkeit, sowie deren Modularer Aufbau.
    Das Fahrgestell unseres Roboters setzt sich aus insgesamt 3 Modulen zusammen.
    Die Motorenhalterung woran die Motoren selbst und auch die H-Brücke befestigt sind.
    Die Zentrale Platform für unseren Microcontroller.
    Und die beiden Freirollen an der Front des Roboters.

    Die gesamte Roboter Platform wurde mit Hilfe eines iterativen design Prozesses gestaltet. 
    Zu Begin wurde nur das Fahrgestell des Roboters designed und gedruckt.

    \begin{figure}[h!]
        \centering
        \includegraphics[width=0.3\textwidth]{imgs/Roboter/CAD/Fahrgestell.jpg}
        \caption{CAD Render vom finalen Fahrgestell}
        \label{fig:cad_fahrgestell}%
    \end{figure}

    Sobald das Ergebniss zufriedenstellend war wurde das nächste Modul designed und gedruckt.
    Dadurch erreichten wir eine Art DEsignkette und konten Sicherstellen das alle Teile bzw. Module zusammen passen
    und allen anforderungen gerecht werden.
    Die letzten beiden Module für unseren Demo Roboter waren die Platform für die Elektronik sowie die Freilaufrollen
    an der Front des Roboters.

    \begin{figure}[h!]
        \centering
        \subfloat[\centering Elektronik Platform (finales Design)]{{\includegraphics[width=0.3\textwidth]{imgs/Roboter/CAD/platform.jpg} }}%
        \qquad
        \subfloat[\centering Freilaufrollen (2. Version)]{{\includegraphics[width=0.3\textwidth]{imgs/Roboter/CAD/follower wheel.jpg} }}%
        \caption{CAD Render der Elektronik Platform und den Freilaufrollen}%
        \label{fig:example}%
    \end{figure}
\end{flushleft}