%Schriftgröße, Layout, Papierformat, Art des Dokumentes
\documentclass[11pt,twoside,a4paper,titlepage]{article}
%Einstellungen der Seitenränder
\usepackage[inner=3.0cm,outer=2.5cm,top=2.5cm,bottom=2.5cm]{geometry}
\usepackage[ngerman]{babel}
\usepackage[utf8]{inputenc}
% wegen "lastchecked"
\usepackage{url}
% hyperref mit sinnvollen einstellungen
\usepackage[pdftex,
        colorlinks=true,
        urlcolor=black,                       % \href{...}{...} external (URL)
        filecolor=black,                      % \href{...} local file
        linkcolor=black,                      % \ref{...} and \pageref{...}
        citecolor=black,
        pdfauthor={Richard Cubek},
        pdfkeywords={},
        pdfproducer={pdfLaTeX},
    	%pdfadjustspacing=1,
        plainpages=false,
        pdfpagelabels,
        pagebackref,
        pdfpagemode=UseOutlines,
        pdfstartview={FitV},
        bookmarksopenlevel=section,
        bookmarksopen=false]{hyperref} 
% meist verwendetes Grafik-Paket
\usepackage[pdftex]{graphicx}
%\usepackage[authoryear]{natbib}
% weitere Pakete
\usepackage{subfig}
\usepackage{amsmath}
\usepackage{amssymb}
\usepackage{setspace}
\usepackage{url}
\usepackage{listings}
\usepackage{algorithm}
\usepackage{algorithmic}
\usepackage{color}
\usepackage{verbatim,framed} 
\usepackage{booktabs}
\usepackage{tabularx}
\usepackage{lipsum}

\onehalfspacing %1.5-facher zeilenabstand

%Kopf- und Fußzeile
\usepackage{fancyhdr}
\fancyhf{}

% Informationen zur Arbeit
\title{{Das ist der Titel meiner Bachelor-Arbeit}}
\author{Erika Musterfrau}
\date{Jan 29th, 2018}

%\setcitestyle{square}

\begin{document}

% no indent
\setlength{\parindent}{0pt}

% Einbinden der Titelseite
\pagestyle{empty}

%\begin{flushright}
%\includegraphics[scale=0.1]{imgs/rwu_logo.png}
%\end{flushright}


\begin{center}


\vspace*{2cm}
\includegraphics[scale=0.15]{imgs/rwu_logo.png}
\vspace*{3cm}

\huge
\textbf{Roboter Orchestrierungssoftware}\\
\Large
\vspace*{2cm}
\noindent \textbf{Expos\'e für die Lehrveranstaltung Systemadministration}\\
\vspace*{0.5cm}
\noindent \textbf{Stefan Geiring, ???}\\
\noindent \textbf{Marvin Müller, 32850}\\
\noindent \textbf{Nicolas Baumgärtner, 32849}\\
\vspace*{0.5cm}
Wintersemester 22/23\\
\normalsize 
05.November 2022
\vspace*{2cm}
\end{center}


\vspace*{4.5cm}
\begin{tabular}{ll}
Betreuer: & M.Sc. Aykan D. Inan \\
 & Ravensburg-Weingarten University of Applied Sciences\\
\end{tabular}



% Seitenstil

%Kopfzeile kapitel außen
%\fancyhead[RO,LE]{Abstract}


\newpage

\pagestyle{fancy}
%\pagestyle{empty}
%\cleardoublepage{}
%\pagestyle{fancy}

% titel/inhaltsverzeichnis

\fancyhead[RO,LE]{Contents}
\tableofcontents

\fancyhead[RO,LE]{\nouppercase{\leftmark}}
\fancyfoot[RO,LE]{\thepage}

%Linie oben
\renewcommand{\headrulewidth}{0.5pt}

% Zähler für Seiten
\setcounter{page}{1}

\newpage

% BEGINN INHALT **************************************************************

\section{Einleitung}

\subsection{Motivation}
\begin{flushleft}
    Die Hauptmotivation für dieses Projekt besteht darin weitere Teile des ROS-Oekosystems neben dem standardmäßigen ROS1 kennenzulernen.
    
    Das bietet die Möglichkeit der Aneignung neuen Wissens in einem fächerübergreifenden Projekt, das sich über 4 Sektoren: der reinen Softwareentwicklung,
    Webentwicklung, Hardware/Elektronik-Entwicklung als auch der Robotik Entwicklung/Forschung erstreckt.\\

    Neben der reinen Aneignung von Wissen soll es uns aber auch die Möglichkeit geben weitere Projekte in Zukunft auf diesem Wissen aufzubauen und gemeinsam mit dem Robotiklabor der Hochschule neue Projekte zu erstellen.
\end{flushleft}

\subsection{Ziel}
\begin{flushleft}
    Ziel des Projektes soll es sein eine
    Orchestrierungssoftware für Roboter auf ROS2 Basis zu entwicklen.
    Mit der Hilfe der Orchestrierungssoftware soll es möglich sein, die Roboter zu überwachen und zu steuern.

    Desweitern soll eine Hardware Platform für einen kleinen Beispiel Roboter entwickelt werden, welche im
    nachhinein durch diverse Komponenten erweitert werden können. Wie zum Beispiel Kamera, Bumper, Laser etc.
\end{flushleft}

\subsection{Eigene Leistung}
\begin{flushleft}

    Ein bedeutender Teil unserer eigenen Leistung ist die Entwickung einer einfachen Weboberfläche auf ROS Basis welche als
    Orchestrierungssoftware für Roboter dienen soll.
    Die Weboberfläche der Orchestrierungssoftware soll in erster Linie als einfache Kontroll- und Debugging-Schnittstele dienen.
    
    Um unsere Orchestrierungssoftware sinngemäß demonstrieren zu können sollen außerdem kleine Roboter auf
    ESP32 Basis gebaut werden. Diese Roboter sollen aus einem 3D-Gedruckten Gehäuse bestehen.
    Auf den ESP32 soll Micro-ROS auf Basis von FreeRTOS ausgeführt werden.

    Unsere kleinen Beispiel-Roboter sollen außerdem im Idealfall mit modular austauschbaren Erweiterungen ausstattbar sein.
    Diese Erweiterungen sollen Simple Sensorik und Aktorik zur Verfügung stellen.
\end{flushleft}

\subsection{Aufbau der Arbeit}
\begin{flushleft}
    Gliederung:
    \begin{enumerate}
    \item Einleitung
    \begin{enumerate}
        \item Motivation
        \item Ziel
        \item Eigene Leistung
        \item Aufbau der Arbeit
    \end{enumerate}
    \item Grundbegriffe (Anhang)
    \item Zielsetzung und Anforderungen
    \item Stand der Technik und Forschung
    \item Lösungsideen
    \item Evaluation der Lösungsideen anhand der Anforderungen
    \item Implementierung
    \item Evaluation der Implementierung
    \item Fazit und Ausblick
    \end{enumerate}
\end{flushleft}

\section{Grundbegriffe (Anhang)}

\section{Zielsetzung und Anforderung}

\section{Stand der Technik und Forschung}
\begin{flushleft}
Es existieren viele Forschungen zu verschiedensten Antriebskinematiken.
Für unser Projekt haben wir uns für einen simplen Diferentialantrieb entschieden. 
\end{flushleft}

\section{Lösungsideen}

\section{Evaluation der Lösungsideen anhand der Anforderungen} 

\section{Implementierung}

\section{Evaluation der Implementierung}

\section{Fazit und Ausblick}


% ENDE INHALT ****************************************************************



%\nocite{*} % list all references
% Stil der Zitate bzw. Verweise
\bibliographystyle{apalike}
%\bibliographystyle{abbrvdin}
%\bibliography{references}
\fancyhead[RO,LE]{References}
%\listoffigures
%\listoftables
\fancyhead[RO,LE]{Listings}
%\listofalgorithms

\end{document}
