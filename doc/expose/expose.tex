%Schriftgröße, Layout, Papierformat, Art des Dokumentes
\documentclass[11pt,twoside,a4paper,titlepage]{article}
%Einstellungen der Seitenränder
\usepackage[inner=3.0cm,outer=2.5cm,top=2.5cm,bottom=2.5cm]{geometry}
\usepackage[ngerman]{babel}
\usepackage[utf8]{inputenc}
% wegen "lastchecked"
\usepackage{url}
% hyperref mit sinnvollen einstellungen
\usepackage[pdftex,
        colorlinks=true,
        urlcolor=black,                       % \href{...}{...} external (URL)
        filecolor=black,                      % \href{...} local file
        linkcolor=black,                      % \ref{...} and \pageref{...}
        citecolor=black,
        pdfauthor={Richard Cubek},
        pdfkeywords={},
        pdfproducer={pdfLaTeX},
    	%pdfadjustspacing=1,
        plainpages=false,
        pdfpagelabels,
        pagebackref,
        pdfpagemode=UseOutlines,
        pdfstartview={FitV},
        bookmarksopenlevel=section,
        bookmarksopen=false]{hyperref} 
% meist verwendetes Grafik-Paket
\usepackage[pdftex]{graphicx}
%\usepackage[authoryear]{natbib}
% weitere Pakete
\usepackage{subfig}
\usepackage{amsmath}
\usepackage{amssymb}
\usepackage{setspace}
\usepackage{url}
\usepackage{listings}
\usepackage{algorithm}
\usepackage{algorithmic}
\usepackage{color}
\usepackage{verbatim,framed} 
\usepackage{booktabs}
\usepackage{tabularx}
\usepackage{lipsum}

\onehalfspacing %1.5-facher zeilenabstand

%Kopf- und Fußzeile
\usepackage{fancyhdr}
\fancyhf{}

% Informationen zur Arbeit
\title{{Das ist der Titel meiner Bachelor-Arbeit}}
\author{Erika Musterfrau}
\date{Jan 29th, 2018}

%\setcitestyle{square}

\begin{document}

% no indent
\setlength{\parindent}{0pt}

% Einbinden der Titelseite
\pagestyle{empty}

%\begin{flushright}
%\includegraphics[scale=0.1]{imgs/rwu_logo.png}
%\end{flushright}


\begin{center}


\vspace*{2cm}
\includegraphics[scale=0.15]{imgs/rwu_logo.png}
\vspace*{3cm}

\huge
\textbf{Roboter Orchestrierungssoftware}\\
\Large
\vspace*{2cm}
\noindent \textbf{Expos\'e für die Lehrveranstaltung Systemadministration}\\
\vspace*{0.5cm}
\noindent \textbf{Stefan Geiring, ???}\\
\noindent \textbf{Marvin Müller, 32850}\\
\noindent \textbf{Nicolas Baumgärtner, 32849}\\
\vspace*{0.5cm}
Wintersemester 22/23\\
\normalsize 
05.November 2022
\vspace*{2cm}
\end{center}


\vspace*{4.5cm}
\begin{tabular}{ll}
Betreuer: & M.Sc. Aykan D. Inan \\
 & Ravensburg-Weingarten University of Applied Sciences\\
\end{tabular}



% Seitenstil

%Kopfzeile kapitel außen
%\fancyhead[RO,LE]{Abstract}


\newpage

\pagestyle{fancy}
%\pagestyle{empty}
%\cleardoublepage{}
%\pagestyle{fancy}

% titel/inhaltsverzeichnis

\fancyhead[RO,LE]{Contents}
\tableofcontents

\fancyhead[RO,LE]{\nouppercase{\leftmark}}
\fancyfoot[RO,LE]{\thepage}

%Linie oben
\renewcommand{\headrulewidth}{0.5pt}

% Zähler für Seiten
\setcounter{page}{1}

\newpage

% BEGINN INHALT **************************************************************

% \section{Einleitung}

\section{Thema}
\begin{flushleft}
    Entwicklung einer Orchestrierungssoftware mit Weboberfläche auf Basis von ROS.
    ROS = Robot Operating System

    Roboter werden immer häufiger eingesetzt egal ob in Spezial Industrial bereichen oder auch einfach zuhause.
    Die einfache und übersichtliche Orchestrierung vieler Roboter ist deshalb besonders wichtig.
    Roboter Schwärme spielen außerdem, in der heutigen Zeit immer öfters eine wichtige Rolle.
    Ein Beispiel für große Roboter Schwärme die bereits eingesetzt werden sind Lieferdrohen, Drohnenshows als 
    ersatz für Feuerwerk oder die simplere Variante davon, eine Lagerverwaltung.
\end{flushleft}

\section{Motivation}
\begin{flushleft}
    Die Hauptmotivation für dieses Projekt besteht darin weitere Teile des ROS-Oekosystems neben dem standardmäßigen ROS1 kennenzulernen.
    
    Das bietet die Möglichkeit der Aneignung neuen Wissens in einem fächerübergreifenden Projekt, das sich über 4 Sektoren: der reinen Softwareentwicklung,
    Webentwicklung, Hardware/Elektronik-Entwicklung als auch der Robotik Entwicklung/Forschung erstreckt.\\

    Neben der reinen Aneignung von Wissen soll es uns aber auch die Möglichkeit geben weitere Projekte in Zukunft auf diesem Wissen aufzubauen und gemeinsam mit dem Robotiklabor der Hochschule neue Projekte zu erstellen.
\end{flushleft}

\section{Ziel}
\begin{flushleft}
    Ziel des Projektes soll es sein eine
    Orchestrierungssoftware für Roboter auf ROS2 Basis zu entwicklen.
    Mit der Hilfe der Orchestrierungssoftware soll es möglich sein, die Roboter zu überwachen und zu steuern.

    Desweitern soll eine Hardware Platform für einen kleinen Beispiel Roboter entwickelt werden, welche im
    nachhinein durch diverse Komponenten erweitert werden können. Wie zum Beispiel Kamera, Bumper, Laser etc.
\end{flushleft}

\section{Eigene Leistung}
\begin{flushleft}

    Ein bedeutender Teil unserer eigenen Leistung ist die Entwickung einer einfachen Weboberfläche auf ROS Basis welche als
    Orchestrierungssoftware für Roboter dienen soll.
    Die Weboberfläche der Orchestrierungssoftware soll in erster Linie als einfache Kontroll- und Debugging-Schnittstele dienen.
    
    Um unsere Orchestrierungssoftware sinngemäß demonstrieren zu können sollen außerdem kleine Roboter auf
    ESP32 Basis gebaut werden. Diese Roboter sollen aus einem 3D-Gedruckten Gehäuse bestehen.
    Auf den ESP32 soll Micro-ROS auf Basis von FreeRTOS ausgeführt werden.

    Unsere kleinen Beispiel-Roboter sollen außerdem im Idealfall mit modular austauschbaren Erweiterungen ausstattbar sein.
    Diese Erweiterungen sollen Simple Sensorik und Aktorik zur Verfügung stellen.
\end{flushleft}

\section{Aufbau der Arbeit}
\begin{flushleft}
    Gliederung:
    \begin{enumerate}
    \item Einleitung
    \begin{enumerate}
        \item Motivation
        \item Ziel
        \item Eigene Leistung
        \item Aufbau der Arbeit
    \end{enumerate}
    \item Grundbegriffe (Anhang)
    \item Zielsetzung und Anforderungen
    \item Stand der Technik und Forschung
    \item Lösungsideen
    \item Evaluation der Lösungsideen anhand der Anforderungen
    \item Implementierung
    \item Evaluation der Implementierung
    \item Fazit und Ausblick
    \end{enumerate}
\end{flushleft}

\section{Stand der Technik und Forschung}
\begin{flushleft}
Es existieren viele Forschungen zu verschiedensten Antriebskinematiken.
Für unser Projekt haben wir uns für einen simplen Diferentialantrieb entschieden. 
\end{flushleft}

\section{React Front-End}
\subsection{Wahl des Frameworks}
\begin{flushleft}
    Da es unglaublich viele Web Frontend Frameworks gibt, die sich leider oft nur in minimalen Punkten unterscheiden,
    haben wir uns Schlussendlich für React aus den unten genannte Gründen entschieden.

    Da Ein Teammitglied bereits viel Erfahrung mit React sammeln konnte sahen wir das als starken Vorteil für unser Team an.
    Außerdem verfügt React über eine enorm gute Dokumentation, da es ständig von seiner eigenen Community verbessert wird.
\end{flushleft}

\subsection{Aufbau des Frontends (Architektur)}
\begin{flushleft}

Da zuvor noch keine Vorerfahrungen zum Erstellen von React-Apps bestand, erfolgte hier erst eine Einarbeitung in diverse Grundlagen und Funktionalitäten in  React.
React ist ein Javascript Framework zum Entwickeln von Webseiten und Webanwendungen.
Statt einfachen statischen HTML-Seiten wird hier mit sogenannten Komponenten gearbeitet, die mehrfach verwendet werden können.
Ebenso können mit States und Hooks reaktive Single-Page-Applications erstellt werden, die ein re-rendern der jeweiligen Komponenten erlauben, ohne ein Neuladen der ganzen Seite.
Weitere Pakete und Bibliotheken können mit dem Paketmanager npm ebenfalls jederzeit nachinstalliert werden.

So gibt es auch die \textit{roslibjs} Bibliothek im npm-Paketstore.
Installation und Importierung in das aktuelle Projekt wird in [\ref{frontend_install}] ausführlicher Beschrieben.
% \begin{lstlisting}[language=bash]
%     npm install roslib 
% \end{lstlisting}

% wird diese dem Projekt hinzugefügt und kann in folgender Weise inkludiert werden:

% \begin{lstlisting}
%     import ROSLIB from 'roslib';
% \end{lstlisting}

Bezüglich der Architektur, bzw. dem Aufbau der Website, haben wir uns an die standardmäßige Vorgehensweise in React gehalten, in dem man die Seite in Komponenten aufteilt und diese so an mehreren Stellen wieder verwenden kann. Dafür haben wir ein extra Verzeichnis \textit{/components} im \textit{/src} Verzeichnis angelegt, sowie eines mit \textit{/pages} für die jeweiligen Seiten. Diese werden in der \textit{App.js} Datei mit dem \textit{react-router-dom} Paket geroutet.


Wir haben uns für eine linksbündige Navigationsleiste entschieden, da diese mehr zu einem Dashboard und einer Konfigurationsseite passt. Auch hier wurde im ersten Moment nicht viel Wert auf ein besonders ausgefallenes Design gelegt. Eine schwarze Navigationsleiste mit einem aufklappbarem Burgermenü war hier für uns ausreichend. 

\hypertarget{rosboard-target}{Für} den generellen Aufbau und dem Design der Website hatten wir uns im Vorfeld schon Gedanken gemacht, so war es uns auf jeden Fall wichtig eine Übersichtseite mit allen Topics zu haben und von diesen die Inhalte auslesen zu können.
Im laufe der Wochen hatten wir ein Gespräch mit Benjamin Stähle aus dem RoboLab an der RWU, wir erzählten ihm von unserem Vorhaben und er zeigte uns ein ROS-Webdashboard namens \textit{rosboard}.
Diese Plattform hatte quasi die Funktionen, die wir für unsere Webanwendung auch geplant hatten.
Zu diesem Zeitpunkt überlegten wir uns, ob wir von nun an diese verwenden, oder unsere eigene Webanwendung programmierten.
Natürlich hätte es hier schon alle unsere gewünschten Funktionen zur Verfügung gehabt, allerdings entschieden wir uns dafür, einmal diesen Prozess von Grund auf selber zu entwickeln und unsere eigene ROS-Webanwendung zu erstellen.
Wir wollten uns aber vom Design und der Vorgehensweise trotzdem an der vorgestellten Anwendung von \textit{rosboard} orientieren.
Ebenso unterscheidet diese sich auch von der Implementierung, da diese auf ganz anderen Bibliotheken basiert.
\\

\vspace{0.5cm}
Um nun die Vorteile von React in unserer Anwendung sinnvoll zu nutzen, unterteilten wir das Connection-Handling, Publishen, Subscriben oder auch beispielsweise die Auflistung der aktuellen Topics, in eigene Komponenten auf.
So konnten wir mit Hilfe von useState-Hooks und useContext-Hooks, eine Singlepage-Application bauen, die mehrere Verbindungen zu ROS-Instanzen verwalten kann.
Für jede offene Verbindung wird ein neues ROSLIB-Objekt angelegt und in ein Array gespeichert.
So kann für jedes dieser Objekte eine neue eigene Unterseite erstellt werden, um mit den ROS-Topics zu interagieren. Diese werden ebenso im Dashboard in einer Tabelle ausgegeben (siehe Abbildung \ref{fig:ros_conn}).

\begin{figure}[h!]
    \centering
    \includegraphics[width=0.8\textwidth]{imgs/web/ros_conn.png}
    \caption{Connection Handler mit Liste und automatischem Navigationspunkt}
    \label{fig:ros_conn}%
\end{figure}

Auf diesen nun generierten Topics-Unterseiten des jeweiligen Roboters, können nun Funktionen darauf angewendet werden.
Für das Publishen auf Topics haben wir uns nur auf das \textit{/cmd\_vel} Topic beschränkt, was einen recht simplem Grund hatte.
Denn zum Publishen wird natürlich auch der Message-Type der Nachricht benötigt.
Um dann dynamisch auf jedes beliebige Topic senden zu können, müsste der User ebenso, auch den entsprechenden Aufbau der Nachricht haben und diese richtig in ein Textfeld eingeben.
Dies wäre sowohl aus User Sicht sehr umständlich und fehleranfällig, da die einzelnen Werte nicht mehr in simplen vordefinierten Textfeldern, sondern in freien Textareas, hätten abgefragt werden müssen. 
Da das \textit{/cmd\_vel} Topic, welches für die Fortbewegung und Lenkung des Roboters zuständig ist, unser definitiv wichtigstes Topic war, beließen wir es dabei.
Dieser Nachrichtentyp enthält drei linear- und drei angular-Werte:

\label{sec:twistMessage}
\begin{lstlisting}
    twist = {
        linear: {
            x: 0,
            y: 0,
            z: 0
        },
        angular: {
            x: 0,
            y: 0,
            z: 0
        }
    }
\end{lstlisting}
Wobei für die Fortbewegung bei Robotern, die sich auf dem Boden mit Rädern bewegen, nur die x-Werte in \textit{linear} und \textit{angular} interessant sind.
Hierfür haben wir im Frontend zwei Optionen entwickelt. 
Zum einen war es uns wichtig fest definierte Werte senden zu können, wofür wir uns für einzelne Textfelder pro Wert entschieden.
Diese wurden dann im Code auf das entsprechende Message Format (Twist) gebracht und an die \textit{rosbridge} gesendet.
Links in Abbildung \ref{fig:web_publish} sind die Textfelder, mit den jeweiligen Buttons zum Senden und Rücksetzen der Nachricht.
Bei Reset wird einfach eine neue Nachricht geschickt, die alle Werte wieder auf Null setzt.

\begin{figure}[h!]
    \centering
    \includegraphics[width=0.8\textwidth]{imgs/web/web_publish.png}
    \caption{Zwei Möglichkeiten um auf \textit{/cmd\_vel} zu publishen}
    \label{fig:web_publish}%
\end{figure}

Ebenso dachten wir uns auch, dass es eine gute Funktionalität wäre, wenn man den Roboter über Tastatur-Eingabe steuern könnte.
So erstellten wir noch ein Textfeld, welches beim Fokussieren die Tastenfelder einließt.
Dies konnte mit dem Event-Handler \textit{onKeyPress} in React umgesetzt werden.
Durch Gedrückthalten der Tasten W, A, S, D, wird der entsprechende linear- oder angular- x-Wert inkrementiert, bzw. dekrementiert.
Mit der Taste R wird auch diese Nachricht wieder zurückgesetzt.
In Abbildung \ref{fig:web_publish} ist dies auf der rechten Seite zu sehen.
Im Anhang in [\ref{webroscomm_code}] wird einmal kurz der Vorgang des Publishen und Subscribens an einem Codebeispiel gezeigt.
\\

\vspace{0.5cm}
Als letztes soll noch auf das Subscriben zu offenen Topics über die Webanwendung beschrieben werden.
Über den Button "List Topics", werden alle momentan zur Verfügung stehenden Topics angeordnet. 
Wie in Abbildung \ref{fig:web_subscribe} zu sehen ist, öffnet sich unterhalt der Topic-Liste ein Fenster für das jeweilige Topics. 
Bei mehreren werden diese nebenbei oder darunter automatisch aufgelistet. 
Innerhalb des Festers werden alle neu ankommenden Nachrichten im JSON-Format gelistet.

\begin{figure}[h!]
    \centering
    \includegraphics[width=0.8\textwidth]{imgs/web/web_subscribe.png}
    \caption{Topics Liste und Subscription zu \textit{/cmd\_vel} Topic}
    \label{fig:web_subscribe}%
\end{figure}

\end{flushleft}
    

\section{ROS Back-End}
\subsection{Warum ROS?}
\begin{flushleft}
    Da wir alle bereits mit ROS gearbeitet haben, ist die Softwareentwicklung für einen Microcontroller deutlich einfacher, da ROS viele Standard-Aufgaben wie die Interprozesskomunikation erledigt. Außerdem ist es für uns leichter bekannte Hürden zu umgehen.
    ROS stellt auch eine große Auswahl an Bibliotheken und Tools bereit, die die Entwicklung ebenfalls erleichtern und beschleunigen.\\
    
    Ein weiterer Grund dafür, dass wir uns für die Entwicklung mit ROS entschieden haben ist, dass wir im Robotiklabor viel Hilfe bekommen können, da dort alle Roboter mit ROS entwickelt werden.
    Weitere Vorteile von ROS sind, dass es open-source ist und kostenlos. \cite{ros}

    
\end{flushleft}

\subsection{Micro-ROS und RTOS}
\begin{flushleft}
    Die Entwicklung mit dem Mikrocontroller findet mit Micro-ROS statt, da das die Standard-Bibliothek für die Entwicklung mit Microcontrollern mit ROS2 ist.\\
    Das Besondere an ROS2 und Micro-ROS im Vergleich zu ROS1 ist, dass es ermöglicht ein Real Time Operating System auf dem Microcontroller auszuführen. Ein RTOS wird standardmäßig mitinstalliert, ist aber für uns auch interessant, da wir bisher noch nicht mit RTOS gearbeitet haben und es eine gute Lernmöglichkeit mit beschränkten Aufwand ist.
    Wie ROS selbst, ist Micro-ROS open-source und kostenlos.\\
    Außerdem gibt es eine bereits bestehende Toolchain namens ESP-IDF, die das Cross-Compilieren, Flashen, Monitoring und Debugging erleichtert.

\end{flushleft}

\subsection{Firmware Kompilierungs Toolchain}
\begin{flushleft}


    Um Programme auf dem ESP32 ausführen zu können müssen sie zuvor crosscompilt und auf den Speicher des Microcontrollers geflasht werden.
    Für das Crosscompilen von Programmen, die das Micro-ROS Framework verwenden wird eine Pipeline empfohlen, die aus 4 Teilen besteht.

    1. ROS
    2. Micro-ROS
    3. RTOS
    4. ESP-IDF

    Zu Grunde liegt ersteinmal die ROS2 Compilierungspipeline welche colcon verwendet.
    Darauf aufbauend folgt die Micro-ROS Pipeline. In dieser Pipeline wird die Crosscompilierung für das entsprechende Realtime Operating System übernommen.
    Des weiteren wird noch die ESP-IDF Pipeline verwendet, die für die Crosscompilierung für den ESP32 zuständig ist und mit deren Hilfe die Programme auf den Mikrocontroller geflash werden können.

    Die Pipeline welche gerade beschrieben wurde, wurde mit Hilfe von Docker implementiert.

    Als erstes wird ein neuer User inklusive Home-Verzeichnis und entsprechenden Cgroup Berechtigungen angelegt.
    Für die Berechtigungen ist vor allem die dialout-Gruppe wichtig, da diese Zugriff auf seriellen Ports gibt.

    Die Verwendung eines eigenen Users ohne root-Rechte wird verwendet, 
    da es von Docker als Best-Practice empfohlen wird um ungewollte Veränderungen zu verhindern (TODO Quelle).


    Nach dem Aufsetzen des Users wird die Micro-ROS Pipeline heruntergeladen und mit HIlfe von mitgelieferten Skripten installiert.

    Nach der Installation müssen noch verschiedene Einstellungen vorgenommen werden. So ist es in eine ersten Schritt notwendig das passende RTOS, in diesem Fall "freertos",
    und den Mikrocontroller zu spezifizieren. Anschließend werden Einstellungen wie die Zugangsdaten für das WLAN-Netzwerk gemacht oder (TODO Beispiel).

    Es ist außerdem noch notwendig den host-user zu den cgroups docker und dialout hinzuzufügen, 
    da es ansonsten nicht möglich ist den Container zu starten oder Programme auf den Mikrocontroller zu übertragen.


    Für das erleichterte Ausführen des Containers wird eine docker-compose-Datei verwendet.

    Der Container wird mit zwei bind-mounts eingerichtet. Einer für das Verzeichnis, 
    in dem der Code des Roboters ist und ein anderer für das Verzeichnis "/dev" in dem die Ports für das flashen des Mikrocontrollers sind.

    Es werden bind-mounts verwendet um die Daten und Ports, während der Entwicklung ständig aktuell zu halten um den Container nicht immer wieder neu starten zu müssen, wenn sich eine Datei ändert.
    Das vereinfacht die Entwicklung deutlich, da man wenn man den Mikrocontroller an und absteckt nicht den Container neu starten muss.
    
\end{flushleft}


\section{3D gedruckte Roboter Platform}
\subsection{Prototyp und Designenscheidungen}
\subsection{Fehler und Verbesserungen}


\end{document}
