\begin{flushleft}
    \begin{description}
        \item[Docker:]\hfill\\
        Software für die Container Verwaltung.

        \item[ROS:]\hfill\\
        Das Acronym ROS steht für Robot Operating System, es bietet eine Sammlung von Software-Bibliotheken, Werkzeugen und ein Framework um die Softwareentwicklung und Ausführung von Andwendugen für Roboter zu erleichtern. \cite{ros}

        \item[React:]\hfill\\
        Web Frontend Framework was ursprünglich von Facebook ins leben gerufen wurde um deklarativ und Komponentenbasiert User-Interfaces zu entwickeln. \cite{reactframework}

        \item[Arduino:]\hfill\\
        Entwicklungsplatform auf Basis von Atmel AtMega Prozessoren. Wurde entworfen um Leien den Einstieg in die Microcontroller
        Welt stark zu vereinfachen. Findet heutzutage weltweit Anwendung in der Maker Szene. \cite{arduino}

        \item[ESP32:]\hfill\\
        ESP32 bezeichnet eine Microntroller Familie auf Basis der ARM Architektur.
        Die ESP32-Familie wurde von der Firma Espressif entwickelt. \cite{esp32}

        \item[RaspberryPi:]\hfill\\
        Single-board Computer der Raspberry Foundation. Der Computer wurde entwickelt um vor allem Kindern das Programmieren und Arbeiten mit Computern näher zu bringen.
        Raspberry pis werden außerdem häufig für Hobby-IOT und Robotikprojekte auf Grund der geringen Kosten im Verhätnis zur Konnektivität und Rechenleistung.\cite{raspberry_pi}
        
        \item[ESP-IDF:]\hfill\\
        ESP-IDF steht für Espressif IOT Development Framework. Das ist ein Framework mit dessen Hilfe man ESP-Socs programmieren kann. \cite{esp_idf}
        
        \item[RTOS]\hfill\\
        Abkürzung für Real Time Operating System. 
        In diesem Projekt wurde in erster Linie das RTOS 'FreeRTOS' verwendet, das für die Verwendung mit Mikrocontrollern optimiert ist. \cite{freertos}
        
        \item[Micro-ROS:]\hfill\\
        Micro-ROS wird ein Framework genannt welches es ermöglicht Mikrocontroller in das ROS Oekosystem einzubinden.
        Mehr dazu im Kapitel Stand der Technik und Forschung. (TODO: LINK?)

        \item[Toolchain:] \hfill\\
        Eine Toolchain ist eine Kombination mehrerer Softwaretools um komplexe Aufgaben in der Software Entwicklung durchzuführen.
        In speziellen Fall dieser Arbeit ist es eine Kombination von Micro-ROS und ESP-IDF.
        Mit Hilfe dieser Tools wird die Software für den Roboter cross-compiliert, geflasht und konfiguriert. \cite{micro_ros}

        \end{description}
\end{flushleft}