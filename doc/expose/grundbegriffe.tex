\begin{flushleft}
    \begin{description}
        \item[Docker:]\hfill\\
        Software für die Container Verwaltung.

        \item[ROS:]\hfill\\
        Das Acronym ROS steht für Robot Operating System, es bietet eine art Framework um Roboter leichter
        einbinden und steuern zu können.

        Der Kern des Frameworks besteht aus Interfaces, genannt ROS-Graphen, die eine anonymisierte und standardisierte Interprozesskommunikation ermöglicht.
        Dieser Graph ist ein Netzwerk aus "Nodes", welche über "Topics" miteinander kommunizieren.
        Auf einem topic wird immer dieselbe "message" von Nodes verbreitet. 
        Für das Verbreiten und Empfangen von Messages müssen die Nodes "Publisher" und "Subscriber" implementieren.

        Des weiteren gibt es noch viele Tools, die beispielsweise Daten visualisieren können, und eine große Menge an Bibliotheken, welche Standard-Algorithmen der Robotik implementieren.
        Diese Tools und Bibliotheken werden ebenfalls zu ROS gezählt weshalb ROS als ehr als ein Framework angesehen wird und den Titel "Operating-System" erhalten hat.

        Es gibt eine ältere version von ROS die einfach nur ROS genannt wird und eine neuere Version namens ROS2.
        Der Unterschied zwischen ROS und ROS2 ist, dass ROS einen zentrallen Server, genannt "ROS-Master" verwendet über den die Kommunikation abläuft und ROS2 einen dezentralen Ansatz verfolgt.
        
        ROS2 baut auf dem Data-Distribution Service Standard von OMG auf. 
        Das ist eine Spezifikation für eine Middleware, welche ein "Data-Centric Subscriber-Publisher (DCPS)"-Modell beschreibt.


        \item[React:]\hfill\\
        Web Frontend Framework was ursprünglich von Facebook ins leben gerufen wurde.

        \item[Arduino:]\hfill\\
        Entwicklungsplatform auf Basis von Atmel AtMega Prozessoren. Wurde entworfen um Leien den Einstieg in die Microcontroller
        Welt stark zu vereinfachen. Findet heutzutage Weltweit Anwendung in der Maker Szene.

        \item[ESP32:]\hfill\\
        ESP32 bezeichnet eine Microntroller Familie auf Basis der ARM Architektur.
        Ursprünglich wurde der ESP32 von Expresif entworfen und gefertigt.

        \item[RaspberryPi:]\hfill\\
        
        \item[Toolchain:] \hfill\\
        Als Toolchain wird die Kombination von Micro-ROS und ESP-IDF bezeichnet.
        Mit Hilfe dieser beiden Frameworks wurde die Firmware für unseren Roboter programmiert und kompiliert.
        \end{description}
\end{flushleft}