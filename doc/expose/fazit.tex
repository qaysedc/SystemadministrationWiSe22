\begin{flushleft}
    Zusammenfassend kann man sagen, dass wir in diesem Projekt viel Neues lernen konnten und dies auch noch in unterschiedlichen Teilgebieten.
    So konnten wir einmal den kompletten Aufbau eines Roboters von Grund auf umsetzen und uns in die einzelnen Schritte hineinarbeiten. 
    Neben dem Hardware-Aufbau konnten wir unser Wissen auch deutlich in der ROS-Umgebung vertiefen, was uns für zukünftige Projekte und Arbeiten definitiv zugutekommen lässt, da wir uns auch vorstellen können später einmal im Robotik-Bereich tätig zu sein.

    Die Einarbeitung in React und die damit entstandene Webanwendung war ebenfalls sehr spannend, da es echt ein tolles Framework zum Erstellen von Single-Page-Applications ist.    
    Um die Anwendung aber als kompletten Ersatz für die reine Terminal-Benutzung von ROS zu ersetzen, würde dies noch einiges an Aufwand kosten.

    Als wichtigsten Punkt zum erwähnen ist jedoch die Verwendung von microROS auf dem ESP32, da dies viele eventuelle spätere Optionen offen hält, Roboter mit einem sehr günstigen Mikrocontroller zu betreiben.
    Daher freuen wir uns auf weitere Projekte innerhalb dieses Bereichs.
\end{flushleft}