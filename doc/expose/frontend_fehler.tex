\begin{flushleft}
    Am Anfang des Projektes planten wir neben dem Publishen und Subscriben der Topics, sowie einer Auflistung der aktuell offenen Topics, noch eine interaktive Konsole.
    Dies umzusetzen stellte sich aber schwieriger als gedacht heraus. 
    Es gab zwar gewisse Tools wie \textit{Xterm.js}, die so etwas ermöglichen sollten, was aber vor allem in Verbindung mit React nicht zum laufen gebracht werden konnte.
    Auch gäbe es einige Node.js Funktionalitäten, um Terminal-Befehle ausführen zu können, doch auch diese stehen unter der Verwendung von React nicht zur Verfügung, da es sich hier um eine isolierte Umgebung handelt und somit ein Zugriff auf OS System Prozesse nicht möglich ist.

    Aus diesem Grund beschlossen wir die interaktive Konsole in der Webanwendung zu streichen.

    Des weiteren hätten wir noch viele spannende Ideen für unsere Webanwendung gehabt.
    Beispielsweise das Speichern von Konfigurations-Einstellungen für einen Roboter.
    Hier wäre auch eine Datenbankanbindung denkbar.
    Ebenso das Publishen auf beliebige Topics, dessen Nachrichten-Typ der Benutzer nicht kennt.
    Sowie das Publishen auf benutzerdefinierte Topics, die in der Webanwendung ebenfalls angelegt hätten angelegt werden können.
    
    Allerdings hat uns dafür deutlich die Zeit für gefehlt, da allein das Einlernen in React anfangs schon viel Zeit in Anspruch genommen hat.
\end{flushleft}