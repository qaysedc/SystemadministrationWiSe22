\begin{flushleft}
    Unsere Arbeit gliedert sich in folgende Teilaspekte auf:
    \begin{itemize}
            \item Evaluation der benötigten Software- und Hardware-Komponenten:
            \begin{itemize}
                \item Mikrocontroller
                \item Antriebe
                \item Frameworks 
            \end{itemize}
            \item Erstellung der ersten Prototypen für:
            \begin{itemize}
                \item die Roboter Platform
                \item das ROS Back-End (ROS in Docker, microROS auf ESP32)
                \item die Webanbindung mit ROS
            \end{itemize}
            \item Programmierung und Implementierung:
            \begin{itemize}
                \item ROS Back-End auf ESP32
                \item Web-Application in React mit Rosbridge und ROSLIBJS
            \end{itemize}
            \item Design des Roboter Gehäuses (3D-Druck)
            \item Entwurf des Roboter-Antriebsstrangs
            \item Zusammenfügen/Aufbau der einzelnen Komponenten
    \end{itemize}

    % \begin{itemize}
    % \item ROS in Docker Container lauffähig bekommen.
    % \item Micro-ROS auf ESP32 oder anderer Hardware lauffähig bekommen.
    % Hardware ist noch nicht final definiert, deshalb muss erst noch evaluiert werden ob der ESP32 unseren
    % Anforderungen und Wünschen gerecht wird.
    % \item Frontend für ROS auf Web-Basis programmieren.
    % \item Evaluation des Frameworks für Web-Frontend (evtl. Software zwischenlayer nötig).
    % \item Design des Roboter Gehäuses entwerfen, welches 3D-Gedruckt wird.
    % \item Antriebsstrang des Roboters entwerfen.
    % \item Evaluation und Testing welcher Antrieb unseren Anforderungen entspricht.
    % \end{itemize}
\end{flushleft}