\begin{flushleft}
Es existieren bereits vielerlei unterschiedlicher fahrbarer Roboter.
Open Source Lösungen für Bastler und Enthuisiasten, Roboter für den Industrial Bereich
\end{flushleft}

\subsection{Software}
Eine der bekanntesten und weitverbreitesten Kontroll Software für Roboter allgemein ist ROS.
ROS steht für "Robot Operating System".

\subsection{Hardware}
Arduino

STM32

ESP32

RaspberryPi

\subsection{Antriebsarten, Vor- und Nachteile}

\begin{flushleft}
Zentraler Einrad Antrieb

Bekannt vom Auto Scooter, an der Front sitzt ein Rad, welches gleichzeitg 
als Antrieb dient, auf einer vertikal drehbaren Achse.
Richtungsvorgabe sowie Vortrieb werden von einer Komponente übernommen.
\end{flushleft}

\begin{flushleft}
Ackerman Lenkung

Jeder der einen Führerschein besitzt kommt mit dieser Art der Lenkung in Berührung.
Heutzutage besitzt jedes Auto diese Art von Lenkung, natürlich in einer stark verbesserten Version.
Die Ackerman Lenkung bietet den Vorteil das Sie den innen und den äßeren Kurvenradius der Reifen berücksichtigt.
\end{flushleft}

\begin{flushleft}
Diferentialantrieb

Vorwiegend verbaut in Baggern und Panzern. Ermöglicht Drehungen auf einem Punkt und kommt ebenfalls 
mit nur zwei Antrieben aus. Eine komplizierte Lenkmechanik wie bei der Ackerman Lenkung ist nicht nötig.
\end{flushleft}

\begin{flushleft}
Mecanum Wheels

Benötigt mindestens vier star verbaute Antriebsmotoren. Vereint Bewegungsmöglichkeiten des Diferentialantriebs
und der Ackerman Lenkung sowie zusätzlich die Lineare seitliche Bewegung.
\end{flushleft}

\begin{flushleft}
Omni Wheels (Poly Wheels)

Ein Vorgänger des Mecanum Wheels welche ebenfalls die gleichen Bewegungsmöglichkeiten bietet.
Es gibt allerdings einen Unterschied: bei einem Omni Wheel Antrieb werden lediglich mindestens drei Antriebe benötigt werden.
\end{flushleft}