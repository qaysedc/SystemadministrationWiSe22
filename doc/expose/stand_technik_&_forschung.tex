\begin{flushleft}
    \textit{Technologischer Standpunkt Software:}\\
    WEBROS TODO
    ROS bietet bereits ein starkes und recht einfach nutzbares Framework, allerdings vermisst man eine 
    schöne und einfach bedienbare "graphische" Oberfläche.
    Es existieren kleinere Projekte welche sich dieser Frontend entwicklung annehmen. 
    Allerdings hat uns keines dieser Projekte zufrieden gestellt. 
    Vor allem was die Orchestrierungsmöglichkeit mehrerer Roboter angeht.
\end{flushleft}

\begin{flushleft}
    \textit{Technologischer Standpunkt Hardware:} \\
    Es existieren bereits viele Forschungen und Beispiele für diverse Roboter
    und deren verschiedensten Antriebskinematiken.
    Für unser Projekt werden wir allerdings eine eigene Roboter Platform designen und 3D-Drucken.
    Viele der bereits vorhandenen Roboter sind entweder zu Groß und teuer oder viel zu klein und deshalb ebenfalls nicht 
    gut für eine demonstration geeignet. Unter anderem wollen wir das unser Roboter den Vorteil bietet weitestgehend
    3D-gedruckt zu sein.
    Die ETH Zürich hat bereits eine kleine Zusammenfassung über die wichtigsten Antriebarten mobiler Roboter zur verfügung gestellt.
    Für unser Projekt haben wir uns vorerst für einen simplen Diferentialantrieb entschieden. 
\end{flushleft}