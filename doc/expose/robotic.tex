\begin{flushleft}
    Ganz klar werden hier die Begriffe \textit{Roboter} und \textit{Robotik} voneinander getrennt.
    Wo der Begriff \textit{Roboter} klar definiert ist und unter strengen Richtlinien und Normen steht, die beschreiben, was ein Roboter ist, da ist der Begriff der \textit{Robotik} nicht genau definiert.
    
    Jedoch versteht man in der Robotik die Steuerung und Anwendung von Robotersystemen.
    Ganz gleich können hier Industrie- und Serviceroboter mit in Verbindung gebracht werden.
    Ein wichtiges Merkmal hierfür ist die Interaktion mit der physischen Welt, wofür ein Roboter Aktoren und Sensoren benutzt.
    Die Robotik ist eine interdisziplinäre Wissenschaft, da sie Gebiete der Informatik, Maschinenbau, Elektronik und der Mathematik vereint. 
    \cite{robotik_konradin}

    Als eines des bekanntesten und meist verbreitetsten Frameworks zur Steuerung von Robotern ist ROS (Robot Operating System), auf das im Folgenden Kapitel der Grundbegriffe genauer eingegangen wird.

\end{flushleft}