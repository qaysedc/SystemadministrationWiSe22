\begin{flushleft}
    Da wir alle bereits mit ROS gearbeitet haben, ist die Softwareentwicklung für einen Microcontroller deutlich einfacher, da ROS viele Standard-Aufgaben wie die Interprozesskomunikation erledigt. Außerdem ist es für uns leichter bekannte Hürden leichter zu umgehen.
    ROS stellt auch eine große Auswahl an Bibliotheken und Tools bereit, die die Entwicklung ebenfalls erleichtern und beschleunigen.\\
    
    Ein weiterer Grund dafür, dass wir uns für die Entwicklung mit ROS entschieden haben ist, dass wir im Robotiklabor viel Hilfe bekommen können, da dort alle Roboter mit ROS entwickelt werden.
    Weitere Vorteile von ROS sind, dass es open-source ist und kostenlos.

    
\end{flushleft}