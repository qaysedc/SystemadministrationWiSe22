\begin{flushleft}

\label{webroscomm_code}Um in der Webanwendung mit ROS kommunizieren zu können sind unsere wichtigsten Funktionen das Subscriben und Publishen von Topics. Im folgenden Codebeispiel wird gezeigt, wie mittels Javascript und der \textit{roslibjs} Bibliothek ein Subscriber erstellt wird und mit einem Listener in einer Callback-Funktion Änderungen aus einem Topic ausgibt.

\begin{lstlisting}
    const my_topic_listener = new ROSLIB.Topic({
        ros,
        name: topicName,
        messageType: msgType,
    });

    my_topic_listener.subscribe((message) => {
        const newTopics = [...alltopicslist, 
                {name: topicName, content: msgType}
            ];
        setTopics(newTopics);
    });
\end{lstlisting}

In \textit{my\_topic\_listener} müssen zur Initialisierung sowohl das ROS-Objekt, der Topic-Name, sowie der Nachrichtentyp des angeforderten Topics enthalten sein.

Als Gegenbeispiel soll hier noch das Publishen von Topics gezeigt werden. Hier haben wir uns erst auf das \textit{/cmd\_vel} Topic beschränkt, welches für die Bewegungsteuerung des Roboters zuständig ist. Hier können lineare Geschwindigkeiten, sowie Einschlagwinkel der Räder übermittelt werden.

\begin{lstlisting}
    const cmd_vel_listener = new ROSLIB.Topic({
        ros : ros,
        name : '/cmd_vel',
        messageType : 'geometry_msgs/Twist'
    });

    var twist = new ROSLIB.Message({
      linear: {
        x: linear,
        y: 0,
        z: 0
      },
      angular: {
        x: 0,
        y: 0,
        z: angular
      }
    });
    cmd_vel_listener.publish(twist);
\end{lstlisting}

\end{flushleft}