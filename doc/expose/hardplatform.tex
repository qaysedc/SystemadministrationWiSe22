\begin{flushleft}
    Die Anforderungen an die Roboter Demo Platform für dieses Projekt waren recht klein und einfach gehalten.

    Das Hauptaugenmerk bei der Neuentwicklung und dem Design unserer Roboter Platform sollte zum einen 
    die 3D-Druckbarkeit des Roboters sein aber auch das der Roboter vor allem in kurzer Zeit gedruckt und betriebsfertig gemacht 
    werden kann.

    In unserem Fall zeigten sich für die Antriebsart beim Differentialantreib die meisten Vorteile.
    Wir benötigen nur 2 Motoren die unabhängig voneinander angesteuert werden müssen. Es können normale
    Reifen verwendet werden und keine aufwendig zu bauende Ketten. Und beim Differentialantreib wird lediglich eine
    um 360° frei rotierbare Rolle benötigt.
    Somit entfällt auch ein großer Aufwand bei der Lenkmechanik. 

    Im allgemeinen sollte die 3D-Druckbarkeit der Teile somit kein Problem darstellen. 

    Für die Elektronik bzw. die Steuerung unseres Roboter fiel die Entscheidung auf einen ESP32. 
    Dieser ist kostengünstig und recht einfach zu beschaffen. 
    Zusätzlich bietet der ESP32 weitaus mehr Rechenleistung als ein Arduino und garantiert somit auch eine gewisse Skalierbarkeit.
    Die restlichen elektronischen Bauteile wie z.B. Lineare Spannungswandler können in diversen Elektronik Versandshops gekauft werden.
    
\end{flushleft}